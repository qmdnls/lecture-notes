% --------------------------------------------------------------
% This is all preamble stuff that you don't have to worry about.
% Head down to where it says "Start here"
% --------------------------------------------------------------
 
\documentclass[12pt]{article}
 
\usepackage[margin=0.7in]{geometry} 
\usepackage{amsmath,amsthm,amssymb}

% times new roman
%\usepackage{newtxtext,newtxmath}

% baskerville
\usepackage{Baskervaldx}
\usepackage[baskervaldx]{newtxmath} 
 
\newcommand{\N}{\mathbb{N}}
\newcommand{\Z}{\mathbb{Z}}
 
\newenvironment{theorem}[2][Theorem]{\begin{trivlist}
\item[\hskip \labelsep {\bfseries #1}\hskip \labelsep {\bfseries #2.}]}{\end{trivlist}}
\newenvironment{lemma}[2][Lemma]{\begin{trivlist}
\item[\hskip \labelsep {\bfseries #1}\hskip \labelsep {\bfseries #2.}]}{\end{trivlist}}
\newenvironment{exercise}[2][Exercise]{\begin{trivlist}
\item[\hskip \labelsep {\bfseries #1}\hskip \labelsep {\bfseries #2.}]}{\end{trivlist}}
\newenvironment{problem}[2][Problem]{\begin{trivlist}
\item[\hskip \labelsep {\bfseries #1}\hskip \labelsep {\bfseries #2.}]}{\end{trivlist}}
\newenvironment{question}[2][Question]{\begin{trivlist}
\item[\hskip \labelsep {\bfseries #1}\hskip \labelsep {\bfseries #2.}]}{\end{trivlist}}
\newenvironment{corollary}[2][Corollary]{\begin{trivlist}
\item[\hskip \labelsep {\bfseries #1}\hskip \labelsep {\bfseries #2.}]}{\end{trivlist}}

\newenvironment{solution}{\begin{proof}[Solution]}{\end{proof}}

% utf-8
\usepackage[utf8]{inputenc}

% cite color
\usepackage[x11names]{xcolor}
\usepackage{hyperref}
\hypersetup{colorlinks=true,%
citecolor=DodgerBlue4,%
filecolor=blue,%
linkcolor=blue,%
urlcolor=blue
}

% line spacing
\renewcommand{\baselinestretch}{1.05}

% margin
\usepackage{geometry}
 \geometry{
 a4paper,
 left=20mm,
 right=20mm,
 top=20mm,
 bottom=20mm
}

% name and student ID in header
\usepackage{fancyhdr}
\pagestyle{fancy}

\fancyhead{}
%\fancyhead[L]{\small Mobile and Ubiquitous Computing}
\fancyhead[C]{Björn Bebensee (2019-21343)}
\fancyhead[R]{March 25, 2020}
\fancypagestyle{plain}{%  the preset of fancyhdr 
    \fancyfoot[C]{\textbf{\thepage}} % except the center
    \renewcommand{\headrulewidth}{0pt}
    \renewcommand{\footrulewidth}{0pt}}

% spacing in itemize environments
\usepackage{enumitem}
\setitemize{noitemsep,topsep=2pt,parsep=2pt,partopsep=2pt}

% misc hyphenation
\hyphenation{page-rank}

% fancyhdr headheight
\setlength{\headheight}{15pt}

% indented paragraph
\newenvironment{indentpar}
    {\vspace{0.25cm}\hfill\begin{minipage}{\dimexpr\textwidth-1cm}}
    {\end{minipage}\vspace{0.3cm}}

\begin{document}
 
% --------------------------------------------------------------
%                         Start here
% --------------------------------------------------------------

% ugly fake title hack
{\Large\centering
    \textbf{Activity Recognition from User-Annotated Acceleration Data}
\par}

\bigskip

\noindent
1. Problem statement: What is the motivation of the paper?

\begin{indentpar}
Bao and Intille (2004) attempt to solve the problem of (physical) activity recognition from data generated by acceleration sensors. As wearable computer systems become more widespread, these can be equipped with such sensors (e.g. wristbands or even adhesive patches) and used to recognize which activity the user is performing. At the time of writing, mobile devices such as smartphones were not very widespread and most previous approaches to solve this problem did not work very well under real-world conditions.
\end{indentpar}


\noindent
2. Solution approach: How did the authors approach the problem?

\begin{indentpar}
Unlike previous work which used researcher-annotated data or focused entirely on laboratory environments, the authors collect user-annotated data from 20 everyday activities in (semi-)naturalistic settings. Data collection was performed while subjects were running an obstacle course. In order to avoid the necessity of direct supervision of a researcher, participants were given freedom in how they performed a task and asked to note start and end times for each obstacle (obstacle sequence). To avoid ambiguity, they also collected data from participants doing the same activities in a more structured setting (activity sequence). Bao and Intille subsequently compute features from the collected data using FFT. With the features they train a decision tree classifier.
\end{indentpar}

\noindent
3. Key contributions of the paper

\begin{indentpar}
One key contribution of the paper is the use of user-annotated data without any intervention by the researches themselves which enables a more natural testing to learn from and to test the developed approach in. Previous work had mostly focused on narrow test settings under constraints or in laboratory environments. They show that user-annotated data allows for collection of a large volume of data that can be leveraged effectively to train a model that works better in real-world conditions.
\end{indentpar}

\noindent
4. Experiments and evaluation

\begin{indentpar}
They evaluate their approach using two different protocols. In the first protocol, they train the classifiers on a specific subject's activity sequence and test in on the same subject's obstacle course sequence. In the second protocol, which they refer to as ``leave-one-subject-out validation'', they train the classifier on both activity sequence and obstacle course data for all but one activity and test it on the remaining activity. The results from the second protocol in particular support their claim that user-annotated data from more naturalistic settings can aid in training a classification model for real-world conditions. However, it is not entirely clear why the authors chose to use these two particular protocols to evaluate their approach.

Interestingly, the authors do not do any testing on subject's that the classifier has not been trained on. For real-world use, a classifier might often be trained on data previously obtained and this data may differ from data that is collected from a new user. To better evaluate the approach, it may be necessary to instead train the classifier on a subset of subjects and test in on the remaining subjects.
\end{indentpar}

\noindent
5. Writing and presentation

\begin{indentpar}
The structure of the paper is good and the writing is very clear. The authors describe their data collection process and their experiments in great detail.
\end{indentpar}

\noindent
6. Literature review

\begin{indentpar}
Literature review is quite extensive. Authors compare their work to several previous works and highlight key differences in their approach.  
\end{indentpar}

\noindent
7. Strong points of the paper

\begin{itemize}
    \item The authors describe all their experiments in great detail
    \item The use of user-annotated data scales much better and allows for training of higher accuracy classifiers 
    \item Review of previous work on the problem is extensive and highlights the significance of this paper
    \item Very detailed evaluation and interpretation of results
\end{itemize}

\noindent
8. Weak points of the paper

\begin{itemize}
    \item It is not clear why they chose these two evaluation protocols
    \item No evaluation on users that the classifier has not been trained on, which might be necessary for real-world use
    \item It does not seem realistic for a subject to wear five (or more) sensors in different body locations in a real-world use case
\end{itemize}

\noindent
9. Conclusion

\begin{indentpar}
Bao and Intille show that accurate activity recognition in a real-world is possible. Their main contribution is a new approach to data collection in which users annotate their own data, allowing for much more data to be collected and used to train a more accurate classifier.
\end{indentpar}

\end{document}
