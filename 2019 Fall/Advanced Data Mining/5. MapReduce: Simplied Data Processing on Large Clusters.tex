% --------------------------------------------------------------
% This is all preamble stuff that you don't have to worry about.
% Head down to where it says "Start here"
% --------------------------------------------------------------
 
\documentclass[12pt]{article}
 
\usepackage[margin=0.7in]{geometry} 
\usepackage{amsmath,amsthm,amssymb}

% times new roman
%\usepackage{newtxtext,newtxmath}

% baskerville
\usepackage{Baskervaldx}
\usepackage[baskervaldx]{newtxmath} 
 
\newcommand{\N}{\mathbb{N}}
\newcommand{\Z}{\mathbb{Z}}
 
\newenvironment{theorem}[2][Theorem]{\begin{trivlist}
\item[\hskip \labelsep {\bfseries #1}\hskip \labelsep {\bfseries #2.}]}{\end{trivlist}}
\newenvironment{lemma}[2][Lemma]{\begin{trivlist}
\item[\hskip \labelsep {\bfseries #1}\hskip \labelsep {\bfseries #2.}]}{\end{trivlist}}
\newenvironment{exercise}[2][Exercise]{\begin{trivlist}
\item[\hskip \labelsep {\bfseries #1}\hskip \labelsep {\bfseries #2.}]}{\end{trivlist}}
\newenvironment{problem}[2][Problem]{\begin{trivlist}
\item[\hskip \labelsep {\bfseries #1}\hskip \labelsep {\bfseries #2.}]}{\end{trivlist}}
\newenvironment{question}[2][Question]{\begin{trivlist}
\item[\hskip \labelsep {\bfseries #1}\hskip \labelsep {\bfseries #2.}]}{\end{trivlist}}
\newenvironment{corollary}[2][Corollary]{\begin{trivlist}
\item[\hskip \labelsep {\bfseries #1}\hskip \labelsep {\bfseries #2.}]}{\end{trivlist}}

\newenvironment{solution}{\begin{proof}[Solution]}{\end{proof}}

% utf-8
\usepackage[utf8]{inputenc}

% cite color
\usepackage[x11names]{xcolor}
\usepackage{hyperref}
\hypersetup{colorlinks=true,%
citecolor=DodgerBlue4,%
filecolor=blue,%
linkcolor=blue,%
urlcolor=blue
}

% line spacing
\renewcommand{\baselinestretch}{1.0}

% margin
\usepackage{geometry}
 \geometry{
 a4paper,
 left=20mm,
 right=20mm,
 top=20mm,
 bottom=20mm
}

% name and student ID in header
\usepackage{fancyhdr}
\pagestyle{fancy}

\fancyhead{}
\fancyhead[L]{Advanced Data Mining}
\fancyhead[C]{Björn Bebensee (2019-21343)}
\fancyhead[R]{October 9, 2019}
\fancypagestyle{plain}{%  the preset of fancyhdr 
    \fancyfoot[C]{\textbf{\thepage}} % except the center
    \renewcommand{\headrulewidth}{0pt}
    \renewcommand{\footrulewidth}{0pt}}

% spacing in itemize environments
\usepackage{enumitem}
\setitemize{noitemsep,topsep=2pt,parsep=2pt,partopsep=2pt}

% misc hyphenation
\hyphenation{page-rank}

% fancyhdr headheight
\setlength{\headheight}{15pt}

\begin{document}
 
% --------------------------------------------------------------
%                         Start here
% --------------------------------------------------------------

% ugly fake title hack
{\Large\centering
    \textbf{MapReduce: Simplified Data Processing on Large Clusters}
\par}

\bigskip

\noindent
1. What is the problem that the paper wants to solve? Why is it difficult (related works)?

\begin{itemize}
    \item Parallel programming for distributed systems is difficult as the programmer needs to take care of partitioning data, scheduling jobs, inter-machine communication etc.
    \item Programmers without distributed computing experience can't use large resources
\end{itemize}

\noindent
2. What is the solution? What is the main idea?

\begin{itemize}
    \item Idea: provide a high-level abstraction that can be used by programmers to formulate their algorithm in a way that can automatically be distributed to a large cluster of machines
    \item Allows for easy and scalable implementation of many algorithms utilizing all resources
    \item Algorithm is divided in a \emph{map} function that applies an operation on each logical record and a \emph{reduce} function which combines data records that have the same key
\end{itemize}

\noindent
3. What is the result?

\begin{itemize}
    \item Tested for grep and sort algorithm: scales very well, performance similar to best reported result on TeraSort benchmark
    \item System is tolerant to process failures (presumably machine failures too but this is not tested)
\end{itemize}

\noindent
4. What is the main novelty that enabled the solution?

\begin{itemize}
    \item MapReduce is an abstraction based on regular distributed programming, parallel programming enabled the solution
    \item Availability of cheap hardware and gigabit ethernet allowed realization of idea in 2003
\end{itemize}

\noindent
5. What are the good aspects of the paper? Did you learn something from the paper?

\begin{itemize}
    \item Authors explain how it has been used successfully in practice in a broad range of problems
\end{itemize}

\noindent
6. What is the impact of the paper?

\begin{itemize}
    \item Solved a real-world problem and has been tested extensively internally at Google prior to publication, has become very popular and been applied practically many times since publication
    \item Enables programmers with no experience in distributed programming to exploit large resources
    \item Fault tolerance considerations have been made as machines in the real-world often fail
\end{itemize}

\noindent
7. Are there weaknesses/missing parts in the paper? How can you improve it?

\begin{itemize}
    \item No direct comparisons or benchmarks to show the speed of MapReduce compared to regular parallel programming
    \item It is not clear if there is any overhead of the abstraction (compared to parallel programming) and how big it is
    \item Limitations as to which programs can not be expressed in the MapReduce framework are not shown
\end{itemize}

\noindent
8. How can you extend the paper?

\begin{itemize}
    \item Explore limitations of MapReduce: which programs can or can't be written as a MapReduce program?
    \item Reduce overhead for machine communication to achieve higher overall throughput
\end{itemize}

\noindent
9. How can you apply the technique to other data/problems?

\begin{itemize}
    \item Create abstraction of a complicated method in a different field to enable programmers without experience in this particular domain to write programs exploiting its benefits
\end{itemize}

\end{document}
