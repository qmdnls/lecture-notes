% --------------------------------------------------------------
% This is all preamble stuff that you don't have to worry about.
% Head down to where it says "Start here"
% --------------------------------------------------------------
 
\documentclass[12pt]{article}
 
\usepackage[margin=1in]{geometry} 
\usepackage{amsmath,amsthm,amssymb}

% times new roman
%\usepackage{newtxtext,newtxmath}

% baskerville
\usepackage{Baskervaldx}
\usepackage[baskervaldx]{newtxmath} 
 
\newcommand{\N}{\mathbb{N}}
\newcommand{\Z}{\mathbb{Z}}
 
\newenvironment{theorem}[2][Theorem]{\begin{trivlist}
\item[\hskip \labelsep {\bfseries #1}\hskip \labelsep {\bfseries #2.}]}{\end{trivlist}}
\newenvironment{lemma}[2][Lemma]{\begin{trivlist}
\item[\hskip \labelsep {\bfseries #1}\hskip \labelsep {\bfseries #2.}]}{\end{trivlist}}
\newenvironment{exercise}[2][Exercise]{\begin{trivlist}
\item[\hskip \labelsep {\bfseries #1}\hskip \labelsep {\bfseries #2.}]}{\end{trivlist}}
\newenvironment{problem}[2][Problem]{\begin{trivlist}
\item[\hskip \labelsep {\bfseries #1}\hskip \labelsep {\bfseries #2.}]}{\end{trivlist}}
\newenvironment{question}[2][Question]{\begin{trivlist}
\item[\hskip \labelsep {\bfseries #1}\hskip \labelsep {\bfseries #2.}]}{\end{trivlist}}
\newenvironment{corollary}[2][Corollary]{\begin{trivlist}
\item[\hskip \labelsep {\bfseries #1}\hskip \labelsep {\bfseries #2.}]}{\end{trivlist}}

\newenvironment{solution}{\begin{proof}[Solution]}{\end{proof}}

% utf-8
\usepackage[utf8]{inputenc}

% cite color
\usepackage{hyperref}
\hypersetup{colorlinks=true,%
citecolor=black,%
filecolor=black,%
linkcolor=black,%
urlcolor=black
}

% line spacing
\renewcommand{\baselinestretch}{1.25}

\begin{document}
 
% --------------------------------------------------------------
%                         Start here
% --------------------------------------------------------------
 
\title{Emergence of scaling in random networks}
\author{Bj\"orn Bebensee (2019\textendash21343)\\ %replace with your name
Advanced Data Mining}
\date{September 9, 2019}
\maketitle

Complex networks describe systems in many different fields of science. However, at the time of writing the paper science was not able to accurately describe the topology of these complex networks. Although these networks have traditionally been described by the random graph theory by Erdős and Renyi (ER) it has now, with increasing amounts of data available, become apparent that ER theory does not suffice. Both ER theory and the more recent small-world model by Watts and Strogatz (WS) fail to address highly-connected vertices that occur in real-world large networks.

Barabási and Albert~\cite{barabasi} show that large networks follow a power law and become scale-free and propose a new network model which incorporates properties of real-world networks while displaying a high degree of self-organization. Real-world networks exhibit continuous growth and expansion and preferential attachment as well as a large amount of self-organization, neither of which have been addressed by existing network models.

The authors found that complex real world networks are scale-free and follow a power law distribution with $P(k) \sim k^{-y}$ where $P(k)$ describes the probability that that a vertex interacts with $k$ other vertices. To achieve this scale-free property they propose to include growth and preferential attachment in the network model. Growth is modeled through new vertices added at every time step. These vertices are then connected to existing vertices. The probability $\Pi$ that a new vertex is connected to existing vertex $i$ is $\Pi(k_i) = \frac{k_i}{\Sigma_j k_j}$ where $k_i$ is its connectivity. They have analyzed a number of real-world network topologies. The main observation is that real-world networks are not accurately described by previous network models as the resulting network topologies differ from real-world large network topologies. They correctly deduce from this observation that the properties these network models fail to describe are network growth and preferential attachment and that these enable generation of scale-free networks. The authors computed $\gamma$ analytically for their model and explored how to modify their model to fit other exponents different from $\gamma = 3$ that their model used initially as real-world scale-free networks typically follow different distributions.

The paper is very brief but "to-the-point" and succinct. The problem was stated very clearly and the approach chosen to solve the problem was explained very well. As the authors predicted in their conclusion, the paper had a big impact on a variety of fields in which it is necessary to model complex networks that occur in the real-world. The proposed method allows for better modeling of large networks and this better description has enabled researchers to better understand systems in domains where less data is available.

The paper could benefit from an application example of their model to a real-world network and an analysis of this application. For many real-world networks the probability $\Pi(k_i)$ of interaction with a vertex $i$ does not only depend on the connectivity or degree of a vertex but also the distance in the network to these nodes. Additionally new edges are added not only for new vertices but one might add new edges to existing vertices too (i.e. new links on a website in the WWW) or vertices and edges can disappear or be removed in real-world networks. These points are not addressed in the paper but it provides a very valuable method to model real-world networks that has seen great success nonetheless.

With data more widely available today than it was at the time of publication one could study other real-world networks and their topologies to observe if these follow power laws and can be described by the proposed model.


\begin{thebibliography}{9}
\bibitem{barabasi} 
Barabási, Albert-László, and Réka Albert. "Emergence of scaling in random networks." science 286, no. 5439 (1999): 509-512.

\end{thebibliography}
 
\end{document}


